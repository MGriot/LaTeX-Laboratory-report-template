%File con config utili allo stile della relazione, in linea di massima qua non devi toccare nulla
%Nome e Cognome 
\def \nome {Nome}
\def \cognome {Cognome}

%page border and dimension
\geometry{a4paper,top=3cm,bottom=3cm,left=1.5cm,right=1.5cm}

%page style
\pagestyle{fancy}
\fancypagestyle{plain}{} %prima pagina uguale alle altre
\fancyhf{}
\rhead{\nome \ \cognome \\ Data consegna: xxxx}
\lhead{Relazione n° x \\ Classe: XE}
\rfoot{Pag. \thepage \hspace{1pt} di \pageref{LastPage}}
\lfoot{\includegraphics[scale=0.1]{img/GitHub/qrcode_github.com.png} Template \LaTeX \;di M.Griot}
\setlength{\headheight}{22.54448pt}
\setlength{\footskip}{41.84514pt}

%%%%%Colors
\definecolor{red}{RGB}{255,0,0}
\definecolor{orange}{RGB}{255,165,0}
\definecolor{blue}{RGB}{0,0,255}
\definecolor{green}{RGB}{143,206,0}

%%%%%%%%%% ToDo notes
% \usepackage[colorinlistoftodos,prependcaption,textsize=tiny,disable]{todonotes}
\newcommandx{\unsure}[2][1=]{\todo[linecolor=red,backgroundcolor=red!25,bordercolor=red,#1]{#2}}
\newcommandx{\change}[2][1=]{\todo[linecolor=orange,backgroundcolor=orange!25,bordercolor=orange,#1]{#2}}
\newcommandx{\info}[2][1=]{\todo[linecolor=blue,backgroundcolor=blue!25,bordercolor=blue,#1]{#2}}
\newcommandx{\improvement}[2][1=]{\todo[linecolor=green,backgroundcolor=green!25,bordercolor=green,#1]{#2}}
\newcommandx{\thiswillnotshow}[2][1=]{\todo[disable,#1]{#2}}

%per flowchart
\tikzstyle{startstop} = [rectangle, rounded corners, minimum width=3cm, minimum height=1cm,text centered, draw=black, fill=red!30]
\tikzstyle{io} = [trapezium, trapezium left angle=70, trapezium right angle=110, minimum width=3cm, minimum height=1cm, text centered, draw=black, fill=blue!30]
\tikzstyle{process} = [rectangle, minimum width=3cm, minimum height=1cm, text centered, draw=black, fill=orange!30]
\tikzstyle{decision} = [diamond, minimum width=3cm, minimum height=1cm, text centered, draw=black, fill=green!30]
\tikzstyle{arrow} = [thick,->,>=stealth]
\pgfplotsset{compat=1.18}

\usetikzlibrary{shapes.geometric, arrows, calc, patterns, positioning}%per flowchart

%per frasi H e P
\newcommand{\Hphrase}[1]{% 
    \IfEqCase{#1}{%
    %Pericoli Fisici
        {H200}{\textbf{H200} – Esplosivo instabile. [\textit{Cancellata}]}%
        {H201}{\textbf{H201} – Esplosivo; pericolo di esplosione di massa.}%
        {H202}{\textbf{H202} – Esplosivo; grave pericolo di proiezione.}%
        {H203}{\textbf{H203} – Esplosivo; pericolo di incendio, di spostamento d'aria o di proiezione.}%
        {H204}{\textbf{H204} – Pericolo di incendio o di proiezione.}%
        {H205}{\textbf{H205} – Pericolo di esplosione di massa in caso d'incendio.}%
        {H206}{\textbf{H206} – Pericolo d'incendio, di spostamento d'aria o di proiezione.}%
        {H207}{\textbf{H207} – Pericolo di incendio o di proiezione.}%
        {H208}{\textbf{H208} – Pericolo d'incendio.}%
        {H220}{\textbf{H220} – Gas altamente infiammabile.}%
        {H221}{\textbf{H221} – Gas infiammabile.}%
        {H222}{\textbf{H222} – Aerosol altamente infiammabile.}%
        {H223}{\textbf{H223} – Aerosol infiammabile.}%
        {H224}{\textbf{H224} – Liquido e vapori altamente infiammabili.}%
        {H225}{\textbf{H225} – Liquido e vapori facilmente infiammabili.}%
        {H226}{\textbf{H226} – Liquido e vapori infiammabili.}%
        {H227}{\textbf{H227} – Liquido combustibile.}%
        {H228}{\textbf{H228} – Solido infiammabile.}%
        {H229}{\textbf{H229} – Contenitore pressurizzato: può scoppiare se riscaldato.}%
        {H230}{\textbf{H230} – Può esplodere anche in assenza di aria.}%
        {H231}{\textbf{H231} – Può esplodere anche in assenza di aria a pressione e/o temperatura elevata.}%
        {H232}{\textbf{H232} – Spontaneamente infiammabile all'aria.}%
        {H240}{\textbf{H240} – Rischio di esplosione per riscaldamento.}%
        {H241}{\textbf{H241} – Rischio d'incendio o di esplosione per riscaldamento.}%
        {H242}{\textbf{H242} – Rischio d'incendio per riscaldamento.}%
        {H250}{\textbf{H250} – Spontaneamente infiammabile all'aria.}%
        {H251}{\textbf{H251} – Autoriscaldante; può infiammarsi.}%
        {H252}{\textbf{H252} – Autoriscaldante in grandi quantità; può infiammarsi.}%
        {H260}{\textbf{H260} – A contatto con l'acqua libera gas infiammabili che possono infiammarsi spontaneamente.}%
        {H261}{\textbf{H261} – A contatto con l'acqua libera gas infiammabili.}%
        {H270}{\textbf{H270} – Può provocare o aggravare un incendio; comburente.}%
        {H271}{\textbf{H271} – Può provocare un incendio o un'esplosione; molto comburente.}%
        {H272}{\textbf{H272} – Può aggravare un incendio; comburente.}%
        {H280}{\textbf{H280} – Contiene gas sotto pressione; può esplodere se riscaldato.}%
        {H281}{\textbf{H281} – Contiene gas refrigerato; può provocare ustioni o lesioni criogeniche.}%
        {H290}{\textbf{} – Può essere corrosivo per i metalli.}%
    %Pericoli per la Salute
        {H300}{\textbf{H300} – Letale se assimilato.}%
        {H301}{\textbf{H301} – Tossico se ingerito.}%
        {H302}{\textbf{H302} – Nocivo per ingestione.}%
        {H303}{\textbf{H303} – Può essere nocivo in caso di ingestione.}%
        {H304}{\textbf{H304} – Può essere letale in caso di ingestione e di penetrazione nelle vie respiratorie.}%
        {H305}{\textbf{H305} – \'E nocivo in caso di ingestione e di penetrazione nelle vie respiratorie.}%
        {H310}{\textbf{H310} – Letale per contatto con la pelle.}%
        {H311}{\textbf{H311} – Tossico per contatto con la pelle.}%
        {H312}{\textbf{H312} – Nocivo per contatto con la pelle.}%
        {H313}{\textbf{H313} – Può essere nocivo per contatto con la pelle.}%
        {H314}{\textbf{H314} – Provoca gravi ustioni cutanee e gravi lesioni oculari.}%
        {H315}{\textbf{H315} – Provoca irritazione cutanea.}%
        {H316}{\textbf{H316} – Provoca una lieve irritazione cutanea.}%
        {H317}{\textbf{H317} – Può provocare una reazione allergica cutanea.}%
        {H318}{\textbf{H318} – Provoca gravi lesioni oculari.}%
        {H319}{\textbf{H319} – Provoca grave irritazione oculare.}%
        {H320}{\textbf{H320} – Provoca irritazione oculare.}%
        {}{\textbf{} – }%
        {}{\textbf{} – }%
        {}{\textbf{} – }%
        {H341}{\textbf{H341} – Sospettato di provocare alterazioni genetiche.}%
        {H350}{\textbf{H350} – Può provocare il cancro.}%
        {H351}{\textbf{H351} – Sospettato di provocare il cancro.}%
        {H361}{\textbf{H361} – Sospettato di nuocere alla fertilità o al feto.}%
        {H372}{\textbf{H372} – Provoca danni agli organi in caso di esposizione prolungata o ripetuta.}%
        {}{\textbf{} – }%
        {}{\textbf{} – }%
        {}{\textbf{} – }%
        {}{\textbf{} – }%
        {}{\textbf{} – }%
        {}{\textbf{} – }%
        {}{\textbf{} – }%
        {}{\textbf{} – }%
        {}{\textbf{} – }%
        {}{\textbf{} – }%
        {}{\textbf{} – }%
        {}{\textbf{} – }%
        {}{\textbf{} – }%
        {}{\textbf{} – }%
        {}{\textbf{} – }%
        {}{\textbf{} – }%
        {}{\textbf{} – }%
        {}{\textbf{} – }%
        {}{\textbf{} – }%
        {}{\textbf{} – }%
        {}{\textbf{} – }%        
        % you can add more cases here as desired
    }[\PackageError{Hphrase}{Undefined option to Hphrase: #1}]%
}%-

\newcommand{\Pphrase}[1]{% 
    \IfEqCase{#1}{%
    %Consigli di prudenza di carattere generale
        {P101}{\textbf{P101} – In caso di consultazione di un medico, tenere a disposizione il contenitore o l'etichetta del prodotto.}%
        {P102}{\textbf{P102} – Tenere fuori dalla portata dei bambini.}%
        {P103}{\textbf{P103} – Leggere l'etichetta prima dell'uso.}%
        {}{\textbf{} – }%
        {P201}{\textbf{P201} – Procurarsi le istruzioni prima dell'uso.}%
        {P202}{\textbf{P202} – Non manipolare prima di avere letto e compreso tutte le avvertenze.}%
        {P260}{\textbf{P260} – Non respirare la polvere/i fumi/i gas/la nebbia/i vapori/gli aerosol.}%
        {P261}{\textbf{P261} – Evitare di respirare la polvere/i fumi/i gas/la nebbia/i vapori/gli aerosol. [modificato]}%
        {P264}{\textbf{P264} – Lavare accuratamente … dopo l’uso.}%
        {P270}{\textbf{P270} – Non mangiare, né bere, né fumare durante l'uso.}%
        {P280}{\textbf{P280} – Indossare guanti/indumenti protettivi/proteggere gli oc­chi/proteggere il viso/pro­teggere l'udito/... [modificato]}%
        {P281}{\textbf{P281} – [soppresso]}%
        {P305 + P351 + P338}{\textbf{P305 + P351 + P338} –  IN CASO DI CONTATTO CON GLI OCCHI: sciacquare accuratamente per parecchi minuti. Togliere le eventuali lenti a contatto se è agevole farlo. Continuare a sciacquare.}%
        {P302 + P352}{\textbf{P302 + P352} – IN CASO DI CONTATTO CON LA PELLE: Lavare abbondantemente con acqua/… [modificato]}%
        {}{\textbf{} – }%
        {}{\textbf{} – }%
        {}{\textbf{} – }%
        {}{\textbf{} – }%
        {}{\textbf{} – }%
        {}{\textbf{} – }%
        {}{\textbf{} – }%
        {}{\textbf{} – }%
        {}{\textbf{} – }%
        {}{\textbf{} – }%
        {}{\textbf{} – }%
        {}{\textbf{} – }%
        {}{\textbf{} – }%
        {}{\textbf{} – }%
        {}{\textbf{} – }%
        {}{\textbf{} – }%
        {}{\textbf{} – }%
        {}{\textbf{} – }%
        {}{\textbf{} – }%
        {}{\textbf{} – }%
        {}{\textbf{} – }%
        {}{\textbf{} – }%
        {}{\textbf{} – }%
        {}{\textbf{} – }%
        {}{\textbf{} – }%
        {}{\textbf{} – }%
        {}{\textbf{} – }%
        {}{\textbf{} – }%
        {}{\textbf{} – }%
        {}{\textbf{} – }%
        {}{\textbf{} – }%
        {}{\textbf{} – }%
        {}{\textbf{} – }%
        {}{\textbf{} – }%
        % you can add more cases here as desired
    }[\PackageError{Pphrase}{Undefined option to Pphrase: #1}]%
}%-
% https://it.wikipedia.org/wiki/Indicazioni_di_pericolo_H
% https://pubchem.ncbi.nlm.nih.gov/ghs/
% https://unece.org/transport/standards/transport/dangerous-goods/ghs-rev9-2021
% https://unece.org/sites/default/files/2021-09/GHS_Rev9E_0.pdf-

%set title and author in pdf information
\hypersetup{pdftitle={Template Example}, pdfauthor={\cognome \; \nome}}

%aggiunge il file di bibliografia per fare le citazioni
\addbibresource{bibliography.bib} %Import the bibliography file



