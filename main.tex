\documentclass[a4paper]{article}
\usepackage[utf8]{inputenc}

\usepackage[italian]{babel}%traduzione parti generate atomaticamente
\usepackage{geometry}
\geometry{a4paper,top=3cm,bottom=3cm,left=1.5cm,right=1.5cm} %dimensioni pagina benedetto
\usepackage[fontsize=13pt]{scrextend}
\usepackage{fancyhdr}
\usepackage{indentfirst} %identatione nella prima frase del paragrafo
\usepackage{lastpage}
\usepackage{xcolor}
\usepackage{mdframed}
\usepackage[version=4]{mhchem}%per formule e equazioni chimiche
\usepackage{chemfig} %per formule 2D chimica
\usepackage{modiagram} % molecular orbital diagrams
\usepackage{wrapfig}
\usepackage{pgfplots}
\usepackage{floatrow}
\usepackage{tikz}
\usepackage{subcaption}
\usepackage{longtable} %Multi-page tables
\usepackage{graphicx}
\usepackage{amsthm}
\usepackage{amssymb}
\usepackage{tabularx}
\usepackage{array}
\usetikzlibrary{shapes.geometric, arrows, calc, patterns, positioning}%per flowchart
\usepackage{amsmath}
\usepackage{booktabs}
\usepackage{multicol}
\usepackage[font=scriptsize]{caption}%per didascalia immagini 
\usepackage{xstring} % per creare p e h phrase command
\usepackage{csquotes}
\usepackage[colorinlistoftodos,prependcaption,textsize=tiny]{todonotes}%solo per le note, utile per prendere appunti, se aggiungi "disable" si tolgono tutti dal pdf ma possono rimanere nel testo.
\usepackage{xargs} % Use more than one optional parameter in a new commands
\usepackage[autocite=superscript,style=chem-acs,articletitle,doi,url]{biblatex}%bibliografia
\usepackage{hyperref}%hyperlinks %package utilizzati per il documento
%File con config utili allo stile della relazione, in linea di massima qua non devi toccare nulla

%%%%%Colors
\definecolor{red}{RGB}{255,0,0}
\definecolor{orange}{RGB}{255,165,0}
\definecolor{blue}{RGB}{0,0,255}
\definecolor{green}{RGB}{143,206,0}

%%%%%%%%%% ToDo notes
% \usepackage[colorinlistoftodos,prependcaption,textsize=tiny,disable]{todonotes}
\newcommandx{\unsure}[2][1=]{\todo[linecolor=red,backgroundcolor=red!25,bordercolor=red,#1]{#2}}
\newcommandx{\change}[2][1=]{\todo[linecolor=orange,backgroundcolor=orange!25,bordercolor=orange,#1]{#2}}
\newcommandx{\info}[2][1=]{\todo[linecolor=blue,backgroundcolor=blue!25,bordercolor=blue,#1]{#2}}
\newcommandx{\improvement}[2][1=]{\todo[linecolor=green,backgroundcolor=green!25,bordercolor=green,#1]{#2}}
\newcommandx{\thiswillnotshow}[2][1=]{\todo[disable,#1]{#2}}

%per flowchart
\tikzstyle{startstop} = [rectangle, rounded corners, minimum width=3cm, minimum height=1cm,text centered, draw=black, fill=red!30]
\tikzstyle{io} = [trapezium, trapezium left angle=70, trapezium right angle=110, minimum width=3cm, minimum height=1cm, text centered, draw=black, fill=blue!30]
\tikzstyle{process} = [rectangle, minimum width=3cm, minimum height=1cm, text centered, draw=black, fill=orange!30]
\tikzstyle{decision} = [diamond, minimum width=3cm, minimum height=1cm, text centered, draw=black, fill=green!30]
\tikzstyle{arrow} = [thick,->,>=stealth]
\pgfplotsset{compat=1.18}

%per frasi H e P
\newcommand{\Hphrase}[1]{% 
    \IfEqCase{#1}{%
    %Pericoli Fisici
        {H200}{\textbf{H200} – Esplosivo instabile. [\textit{Cancellata}]}%
        {H201}{\textbf{H201} – Esplosivo; pericolo di esplosione di massa.}%
        {H202}{\textbf{H202} – Esplosivo; grave pericolo di proiezione.}%
        {H203}{\textbf{H203} – Esplosivo; pericolo di incendio, di spostamento d'aria o di proiezione.}%
        {H204}{\textbf{H204} – Pericolo di incendio o di proiezione.}%
        {H205}{\textbf{H205} – Pericolo di esplosione di massa in caso d'incendio.}%
        {H206}{\textbf{H206} – Pericolo d'incendio, di spostamento d'aria o di proiezione.}%
        {H207}{\textbf{H207} – Pericolo di incendio o di proiezione.}%
        {H208}{\textbf{H208} – Pericolo d'incendio.}%
        {H220}{\textbf{H220} – Gas altamente infiammabile.}%
        {H221}{\textbf{H221} – Gas infiammabile.}%
        {H222}{\textbf{H222} – Aerosol altamente infiammabile.}%
        {H223}{\textbf{H223} – Aerosol infiammabile.}%
        {H224}{\textbf{H224} – Liquido e vapori altamente infiammabili.}%
        {H225}{\textbf{H225} – Liquido e vapori facilmente infiammabili.}%
        {H226}{\textbf{H226} – Liquido e vapori infiammabili.}%
        {H227}{\textbf{H227} – Liquido combustibile.}%
        {H228}{\textbf{H228} – Solido infiammabile.}%
        {H229}{\textbf{H229} – Contenitore pressurizzato: può scoppiare se riscaldato.}%
        {H230}{\textbf{H230} – Può esplodere anche in assenza di aria.}%
        {H231}{\textbf{H231} – Può esplodere anche in assenza di aria a pressione e/o temperatura elevata.}%
        {H232}{\textbf{H232} – Spontaneamente infiammabile all'aria.}%
        {H240}{\textbf{H240} – Rischio di esplosione per riscaldamento.}%
        {H241}{\textbf{H241} – Rischio d'incendio o di esplosione per riscaldamento.}%
        {H242}{\textbf{H242} – Rischio d'incendio per riscaldamento.}%
        {H250}{\textbf{H250} – Spontaneamente infiammabile all'aria.}%
        {H251}{\textbf{H251} – Autoriscaldante; può infiammarsi.}%
        {H252}{\textbf{H252} – Autoriscaldante in grandi quantità; può infiammarsi.}%
        {H260}{\textbf{H260} – A contatto con l'acqua libera gas infiammabili che possono infiammarsi spontaneamente.}%
        {H261}{\textbf{H261} – A contatto con l'acqua libera gas infiammabili.}%
        {H270}{\textbf{H270} – Può provocare o aggravare un incendio; comburente.}%
        {H271}{\textbf{H271} – Può provocare un incendio o un'esplosione; molto comburente.}%
        {H272}{\textbf{H272} – Può aggravare un incendio; comburente.}%
        {H280}{\textbf{H280} – Contiene gas sotto pressione; può esplodere se riscaldato.}%
        {H281}{\textbf{H281} – Contiene gas refrigerato; può provocare ustioni o lesioni criogeniche.}%
        {H290}{\textbf{} – Può essere corrosivo per i metalli.}%
    %Pericoli per la Salute
        {H300}{\textbf{H300} – Letale se assimilato.}%
        {H301}{\textbf{H301} – Tossico se ingerito.}%
        {H302}{\textbf{H302} – Nocivo per ingestione.}%
        {H303}{\textbf{H303} – Può essere nocivo in caso di ingestione.}%
        {H304}{\textbf{H304} – Può essere letale in caso di ingestione e di penetrazione nelle vie respiratorie.}%
        {H305}{\textbf{H305} – \'E nocivo in caso di ingestione e di penetrazione nelle vie respiratorie.}%
        {H310}{\textbf{H310} – Letale per contatto con la pelle.}%
        {H311}{\textbf{H311} – Tossico per contatto con la pelle.}%
        {H312}{\textbf{H312} – Nocivo per contatto con la pelle.}%
        {H313}{\textbf{H313} – Può essere nocivo per contatto con la pelle.}%
        {H314}{\textbf{H314} – Provoca gravi ustioni cutanee e gravi lesioni oculari.}%
        {H315}{\textbf{H315} – Provoca irritazione cutanea.}%
        {H316}{\textbf{H316} – Provoca una lieve irritazione cutanea.}%
        {H317}{\textbf{H317} – Può provocare una reazione allergica cutanea.}%
        {H318}{\textbf{H318} – Provoca gravi lesioni oculari.}%
        {H319}{\textbf{H319} – Provoca grave irritazione oculare.}%
        {H320}{\textbf{H320} – Provoca irritazione oculare.}%
        {}{\textbf{} – }%
        {}{\textbf{} – }%
        {}{\textbf{} – }%
        {H341}{\textbf{H341} – Sospettato di provocare alterazioni genetiche.}%
        {H350}{\textbf{H350} – Può provocare il cancro.}%
        {H351}{\textbf{H351} – Sospettato di provocare il cancro.}%
        {H361}{\textbf{H361} – Sospettato di nuocere alla fertilità o al feto.}%
        {H372}{\textbf{H372} – Provoca danni agli organi in caso di esposizione prolungata o ripetuta.}%
        {}{\textbf{} – }%
        {}{\textbf{} – }%
        {}{\textbf{} – }%
        {}{\textbf{} – }%
        {}{\textbf{} – }%
        {}{\textbf{} – }%
        {}{\textbf{} – }%
        {}{\textbf{} – }%
        {}{\textbf{} – }%
        {}{\textbf{} – }%
        {}{\textbf{} – }%
        {}{\textbf{} – }%
        {}{\textbf{} – }%
        {}{\textbf{} – }%
        {}{\textbf{} – }%
        {}{\textbf{} – }%
        {}{\textbf{} – }%
        {}{\textbf{} – }%
        {}{\textbf{} – }%
        {}{\textbf{} – }%
        {}{\textbf{} – }%        
        % you can add more cases here as desired
    }[\PackageError{Hphrase}{Undefined option to Hphrase: #1}]%
}%-

\newcommand{\Pphrase}[1]{% 
    \IfEqCase{#1}{%
    %Consigli di prudenza di carattere generale
        {P101}{\textbf{P101} – In caso di consultazione di un medico, tenere a disposizione il contenitore o l'etichetta del prodotto.}%
        {P102}{\textbf{P102} – Tenere fuori dalla portata dei bambini.}%
        {P103}{\textbf{P103} – Leggere l'etichetta prima dell'uso.}%
        {}{\textbf{} – }%
        {P201}{\textbf{P201} – Procurarsi le istruzioni prima dell'uso.}%
        {P202}{\textbf{P202} – Non manipolare prima di avere letto e compreso tutte le avvertenze.}%
        {P260}{\textbf{P260} – Non respirare la polvere/i fumi/i gas/la nebbia/i vapori/gli aerosol.}%
        {P261}{\textbf{P261} – Evitare di respirare la polvere/i fumi/i gas/la nebbia/i vapori/gli aerosol. [modificato]}%
        {P264}{\textbf{P264} – Lavare accuratamente … dopo l’uso.}%
        {P270}{\textbf{P270} – Non mangiare, né bere, né fumare durante l'uso.}%
        {P280}{\textbf{P280} – Indossare guanti/indumenti protettivi/proteggere gli oc­chi/proteggere il viso/pro­teggere l'udito/... [modificato]}%
        {P281}{\textbf{P281} – [soppresso]}%
        {P305 + P351 + P338}{\textbf{P305 + P351 + P338} –  IN CASO DI CONTATTO CON GLI OCCHI: sciacquare accuratamente per parecchi minuti. Togliere le eventuali lenti a contatto se è agevole farlo. Continuare a sciacquare.}%
        {P302 + P352}{\textbf{P302 + P352} – IN CASO DI CONTATTO CON LA PELLE: Lavare abbondantemente con acqua/… [modificato]}%
        {}{\textbf{} – }%
        {}{\textbf{} – }%
        {}{\textbf{} – }%
        {}{\textbf{} – }%
        {}{\textbf{} – }%
        {}{\textbf{} – }%
        {}{\textbf{} – }%
        {}{\textbf{} – }%
        {}{\textbf{} – }%
        {}{\textbf{} – }%
        {}{\textbf{} – }%
        {}{\textbf{} – }%
        {}{\textbf{} – }%
        {}{\textbf{} – }%
        {}{\textbf{} – }%
        {}{\textbf{} – }%
        {}{\textbf{} – }%
        {}{\textbf{} – }%
        {}{\textbf{} – }%
        {}{\textbf{} – }%
        {}{\textbf{} – }%
        {}{\textbf{} – }%
        {}{\textbf{} – }%
        {}{\textbf{} – }%
        {}{\textbf{} – }%
        {}{\textbf{} – }%
        {}{\textbf{} – }%
        {}{\textbf{} – }%
        {}{\textbf{} – }%
        {}{\textbf{} – }%
        {}{\textbf{} – }%
        {}{\textbf{} – }%
        {}{\textbf{} – }%
        {}{\textbf{} – }%
        % you can add more cases here as desired
    }[\PackageError{Pphrase}{Undefined option to Pphrase: #1}]%
}%-
% https://it.wikipedia.org/wiki/Indicazioni_di_pericolo_H
% https://pubchem.ncbi.nlm.nih.gov/ghs/
% https://unece.org/transport/standards/transport/dangerous-goods/ghs-rev9-2021
% https://unece.org/sites/default/files/2021-09/GHS_Rev9E_0.pdf-

%aggiunge il file di bibliografia per fare le citazioni
\addbibresource{bibliography.bib} %Import the bibliography file % file con settaggi da non toccare

% Nome e Cognome 
\def \nome {Nome}
\def \cognome {Cognome}

%titolo e data
\title{Laboratorio di xxxx \\ \huge{Template relazioni}}
\author{\nome \ \cognome}
\date{\today} %metti la data di quando hai fatto l'esperienza

%page style
\pagestyle{fancy}
\fancypagestyle{plain}{} %prima pagina uguale alle altre
\fancyhf{}
\rhead{\nome \ \cognome \\ Data consegna: xxxx}
\lhead{Relazione n° x \\ Classe: XE}
\rfoot{Pag. \thepage \hspace{1pt} di \pageref{LastPage}}
\setlength{\headheight}{22.54448pt}


\begin{document}

\maketitle %crea il titolo
\noindent\rule{1\textwidth}{0.5pt}
    \begin{abstract}
        La relazione tecnica che conclude un'esperienza ha lo scopo di comunicare gli obbiettivi del proprio lavoro, le modalità con cui si è svolto e i risultati ottenuti. 
        
        Essa dev'essere redatta in modo tale che chiunque possari produrre l'esperimento realizzato e confrontare i risultati.
        
        Per questo motive la relazione tecnica deve essere articolata, nell'ordine, nei seguenti punti.
    \end{abstract} 
\noindent\rule{1\textwidth}{0.5pt} % elimina
\section{Objective}
Sometimes it is referred to as Purpose.

It is useful to express the objectives of the experiment (however, sometimes it is not necessary because it is already indicated in the title).
\info[inline]{example "Purification and recrystallization of benzoic acid after contaminating it with activated carbon."}

\section{Principio del metodo}
A volte è indicato come Sommario o Principio teorico.

Fa riferimento a tutti i principi teorici su cui si basa l'esperienza e a come essi si sono utilizzati per raggiungere gli obbiettivi prefissati. Qui vanno messsi tutti i concetti teorici necessari a comprendere la relazione, bisogna spiegare tutto quello che merita di essere spiegato. Se si sono già scritte delle relazioni, o la relazione fa parte di un macro gruppo di relazioni, si può pensare di omettere alcune parti che potrebbero ripetersi e snellire un po' il tutto.

Un esempio può essere:
\begin{description}
	\item[Stoichiometry] The relationship between the relative quantities of substances taking part in a reaction or forming a compound, typically a ratio of whole integers.
	\item[Atomic mass] The mass of an atom of a chemical element expressed in atomic mass units. It is approximately equivalent to the number of protons and neutrons in the atom (the mass number) or to the average number allowing for the relative abundances of different isotopes. 
\end{description} 

Oppure usare il comando \verb|\subsection{}| per indicare i vari punti.

\section{Strumenti}
A volte è indicato come Apparecchiatura.

Specifica il tipo di vetreria e strumentazione previsti dall'esperienza e riporta tutti  i dati tecnici ritenuti signficativi. 
\info[inline]{es. la tolleranza e la portata della vetreria.}

Vanno messi tutti gli strumenti utilizzati durante l'esperienza o esperimento in maniera minuziosa. Anche il tipo di vetreria (pyrex o no, la classe della vetreria, portata e sensibilità sono importanti). 
In pratica vanno messi i dati tecnici degli strumenti utilizzati. 

Si può usare un elenco puntato e, all'occorrenza, annidarne uno dentro per meglio raggruppare gli strumenti.

es.
\begin{itemize}
    \item Strumenti:
    \begin{itemize}
        \item Vetrino porta oggetti;
        \item Bunsen;
        \item Ansa;
        \item Pinze in legno;
        \item Buretta, portata 25 mL e tolleranza 0.05 mL;
        \item ...
    \end{itemize}
    \item Terreni:
    \begin{itemize}
    \item Malt Agar;
    \item Nutrient Agar.
    \end{itemize}
    \item Coloranti
    \begin{itemize}
        \item Blu di metile;
        \item Violetto di genziana.
    \end{itemize}
\end{itemize}

Quando la lista è particolarmente lunga sarebbe meglio creare due colonne, rende meno papiro la relazione e riempie meglio gli spazi.

es.
\begin{multicols}{2}
\begin{itemize}
    \item Strumenti:
    \begin{itemize}
        \item Vetrino porta oggetti;
        \item Bunsen;
        \item Ansa;
        \item Pinze in legno;
        \item ...
    \end{itemize}
    \item Terreni:
    \begin{itemize}
    \item Malt Agar;
    \item Nutrient Agar.
    \end{itemize}
    \item Coloranti
    \begin{itemize}
        \item Blu di metile;
        \item Violetto di genziana.
    \end{itemize}
\end{itemize}
\end{multicols}

Un'altra alternativa è creare una tabella e mettere all'interno i vari strumenti utilizzati divisi per colonne.

es.
\begin{center}
   \begin{tabular}{l|c|r}
   \toprule
     \textbf{Strumenti} &  \textbf{Terreni} & \textbf{Coloranti}\\
    \midrule
     Vetrino porta oggetti & Malt Agar & Blu di metile\\
     Bunsen & Nutrient Agar & Violetto di genziana\\
     Ansa & & \\
     Pinze in legno & & \\
     ... & & \\
     \bottomrule
    \end{tabular} 
\end{center}
\section{Reagenti}
Indica il tipo e le caratteristiche dei reagenti  impiegati (è essenziale segnalare la concentrazione delle soluzioni e, per i solidi, il grado di purezza).
Nel caso i reagenti disponessero di una scheda di sicurezza è buona norma riportare in maniera sintetica tali informazioni come pittogrammi, frasi H e P, i DPI necessari la TLV.

A questo scopo viene mostrato anche come mettere le equazioni e formule chimiche con un pacchetto apposito:
\begin{itemize}
    \item \ce{H3PO4} 0.3 M 25 mL
\end{itemize}

\begin{table}[!ht]
    \scriptsize
    \centering
    \begin{tabularx}{\textwidth}{m{0.15\textwidth}|m{0.2\textwidth}|m{0.23\textwidth}|m{0.17\textwidth}|m{0.13\textwidth}}
        \toprule
        \textbf{Composto} &  \textbf{Formula di struttura} & \textbf{Frasi H e P} & \textbf{Pittogrammi} & \textbf{DPI}\\
        \midrule
        \ce{H2O2}& \begin{center}\includegraphics[width=3cm,scale=0.4]{img/763.png} \end{center} & 
             H: H271, H302, H314 e H332.
             
             P: P210, P220, P221, P260, P261, P264, P270, P271, P280, P283, P301+P312, P301+P330+P331, P303+P361+P353, P304+P312, P304+P340, P305+P351+P338, P306+P360, P310, P312, P321, P330, P363, P370+P378, P371+P380+P375, P405, e P501. & \begin{center} \begin{tabular}{cc}
                  \includegraphics[scale=0.15]{img/pittogrammi/Flammable.png}&  \includegraphics[scale=0.15]{img/pittogrammi/Explosive.png} \\
                  \includegraphics[scale=0.15]{img/pittogrammi/Flammable.png}&  \includegraphics[scale=0.15]{img/pittogrammi/Explosive.png} 
             \end{tabular}\end{center} & guanti, occhiali e visiera\\
        \bottomrule
    \end{tabularx}
    \caption{Tabella con i composti chimici utilizzati nell'esperienza, le frasi P e H vengono riportate per esteso al fondo della relazione.} % da modificare
    \label{tab:tab1} % da modificare
    \normalsize
\end{table}

\newpage
Tabella precedente con i TLV (threshold limit value) che sono valori di concentrazione di sostanze aerodisperse, più o meno tossiche, al di sotto delle quali la maggior parte dei lavoratori può rimanere esposta ripetutamente tutti giorni senza effetti dannosi per la salute.

\begin{table}[!ht]
    \centering
    \scriptsize
    \begin{tabularx}{1\textwidth}{m{0.15\textwidth}|m{0.15\textwidth}|m{0.2\textwidth}|m{0.12\textwidth}|m{0.1\textwidth}|m{0.13\textwidth}}
        \toprule
        \textbf{Composto} &  \textbf{Formula di struttura} & \textbf{Frasi H e P} & \textbf{Pittogrammi} & \textbf{DPI} & \textbf{TLV [ppm]}\\
        \midrule
        \ce{H2O2}& \begin{center}\includegraphics[width=0.15\textwidth,scale=0.4]{img/763.png} \end{center} & \Hphrase{H271} \Hphrase{H302} \Pphrase{P261} &
                \begin{tabular}{c}
                  \includegraphics[scale=0.15]{img/pittogrammi/Flammable.png} \\ \includegraphics[scale=0.15]{img/pittogrammi/Explosive.png} \\
                  \includegraphics[scale=0.15]{img/pittogrammi/Flammable.png} \\ \includegraphics[scale=0.15]{img/pittogrammi/Explosive.png} 
             \end{tabular}& guanti, occhiali e visiera & LTEL: 100 
             
             STEL: 200 \\
    \bottomrule
    \end{tabularx}
    \caption{Tabella con i composti chimici utilizzati nell'esperienza, le frasi P e H vengono riportate per esteso al fondo della relazione. Ricorda che: Long-term Exposure Limit (LTEL) Values e Short-term Exposure Limit (STEL) Values} % da modificare
    \label{tab:tab2} % da modificare
    \normalsize
\end{table}

Oppure, in alcuni casi, è necessaria una tabella differente come la seguente.

Esprime i rapporti stechiometrici, le masse e le rese, utili nelle reazioni di sintesi quando si deve capire il meccanismo e quest'ultimo può variare in base alle concentrazioni dei reagenti.

\begin{table}[ht]
    \centering
    \scriptsize
    \begin{tabularx}{1\textwidth}{X|X|X|X|X|X|X|X|X}
    \toprule
         Sostanza & Massa molecolare\newline[g/mol] & Moli\newline[mol] & Rapporto stechiometrico & Massa\newline[g] & Volume\newline[mL] & Resa\newline[\%] & Frasi H & Frasi P\\
    \midrule
         A & B & C & D & E & F & G & H & I \\
    \bottomrule
    \end{tabularx}
    \caption{Tabella per sintesi}
    \label{tab:tab3}
    \normalsize
\end{table}

\begin{table}[ht]
    \centering
    \scriptsize
    \begin{tabularx}{1\textwidth}{X|X|X|X|X|X|X}
    \toprule
    Composto & Struttura & Aspetto & Rischi & Protezioni &  TLV/TWA & Smaltimento \\
    \midrule
        \ce{H2O}  &  & liquido incolore & nessuno & nessuna & irrilevante & semplicemente gettare nel lavandino\\
    \midrule
        \ce{H2O}  &  & liquido incolore & nessuno & nessuna & irrilevante & semplicemente gettare nel lavandino\\
    \midrule    
        \ce{H2O}  &  & liquido incolore & nessuno & nessuna & irrilevante & semplicemente gettare nel lavandino\\
    \midrule    
        \ce{H2O}  &  & liquido incolore & nessuno & nessuna & irrilevante & semplicemente gettare nel lavandino\\
    \bottomrule
    \end{tabularx}
    \caption{Altro modello di tabella.}
    \label{tab:my_label}
\end{table}

Per i DPI si è costretti a cercare il nome del composto e scrivere di seguito scheda di sicurezza su un qualisasi motore di ricera e cercando il risultato più recente.

\info[inline]{es. acido benzoico scheda di sicurezza.} 

Ricordarsi il produttore, facendo una foto in laboratorio del contenitore, può accelerare il lavoro ed essere più corretti in quanto le informazioni potrebbero non sempre essere uguali.
In fondo alla secheda si trova, oltre alle informazioni sui DPI, altre varie informazioni utili.

\section{Procedure}
Reports the operating procedure followed. It is often completed with a schematic drawing of the equipment used, when it is useful to describe the assembly instructions of the same.
Here you need to explain in a synthetic but exhaustive way all the steps taken during the experiment.

Comments or observations are also allowed, if they make sense. In the case of details not mentioned in the procedure but observed during the experiment, such as the increase in temperature in a reaction that would prevent holding a test tube in your hand.

\info[inline]{e.g. Add Devarda's alloy. \textit{Warning! After the addition the content of the test tube reaches high temperatures, it is better to carry out the operation near a counter and with a crucible to place the test tube.}}


Another thing you can add is a flowchart that summarizes the steps, this can also be useful to have a clear procedure and possibly study.

\section{Reazioni}
Scrivere le reazione che avengono nell'esperienza

\ce{2H2O^2+}

\ce{H2O <=> 2H+ + 2O-}

\ce{CH#CH}

\ce{CH2=CH2}

\ce{CH3-CH3}

\ce{A ->[C] B}
\section{Dati e Calcoli}
In questa sezione si deve raccogliere tutti i risultati che si sono ottenuti. Vanno bene foto, tabelle di dati e osservazioni personali. I dati sarebbe buono che vengano raccolti in tabella se sono in numero sufficiente. 
\'E anche vero che si possono unire le due sezioni dei dati per rendere più discorsivo il tutto.

In questa sezione si devono anche scrivere tutte le formule che si usano per i calcoli indicando cosa servono e le loro unità di misura.

Ci sono due modi per impostare le cose, nel primo modo si scrivono prima tutte le formule e poi dopo si svolgono i conti con i propri dati, oppure in alternativa, si scrive la formula e poi subito dopo il conto relativo. Se si sceglie di fare il secondo metodo si potrebbe usare una tabella

\textbf{Primo metododo:}
Le moli si trovano tramite la seguente formula:
\begin{equation}
   n=M[mol/L]\cdot V[L]=n [mol] 
   \label{eq:mol} %questo ti serve per citare le equazioni, in questa sezione non è molto utile ma nei cenni teorici ti può aiutare a scrivere meglio le frasi senza intortarti da solo
\end{equation}
E ora si svolge il calcolo sui propri dati, in questo caso aggiungo un "*" all'ambiente per togliere i numeri:
\begin{equation*}
   n=\frac{0.5\cdot 25}{1000}=0.0125 [mol] 
\end{equation*}
Si divide per  1000 percheè il volume è stato espresso in mL.

\textbf{Nel secondo modo invece:}
\begin{center}
    \begin{tabularx}{0.9\textwidth}{XcX}
    Formule & & Calcoli\\
    \midrule
    $n=M[mol/L]\cdot V[L]=n [mol] $ &$\rightarrow$& $ n=\frac{0.5\cdot 25}{1000}=0.0125 [mol] $\\
    $n=M[mol/L]\cdot V[L]=n [mol] $ &$\rightarrow$& $ n=\frac{0.5\cdot 25}{1000}=0.0125 [mol] $\\
    $n=M[mol/L]\cdot V[L]=n [mol] $ &$\rightarrow$& $ n=\frac{0.5\cdot 25}{1000}=0.0125 [mol] $\\
    $n=M[mol/L]\cdot V[L]=n [mol] $ &$\rightarrow$& $ n=\frac{0.5\cdot 25}{1000}=0.0125 [mol] $\\
    $n=M[mol/L]\cdot V[L]=n [mol] $ &$\rightarrow$& $ n=\frac{0.5\cdot 25}{1000}=0.0125 [mol] $\\
    \end{tabularx}
\end{center}
Bisogna trovare il modo per distanziare un po' il testo in verticale ma ci può stare.

\subsection{Dati sperimentali}
Vanno riportati \textit{tutti} i dati sperimentali, evidenziando se necessario, quelli "aberranti", cioè da scartare sulla base si un analisi statistica.
Laddove possibile, è bene raccogliere i dati sotto forma di tabelle.
\vspace{1ex}
\begin {center}
\begin{tabular}{c|c}
     Esperimento &  Risultato\\
     1 & 10\\
     2 & 15\\
     3 & ...
\end{tabular}
\end {center}

\subsection{Elaborazione dei dati}
(Se necessario anche grafica): riporta i calcoli effettuati a partire dai risultati sperimentali, indicando le relazioni matematiche utilizzate. L'elaborazione può consistere anche nella costruzione di diagrammi o grafici.
A volte essa prevede anche il trattamento statistico dei dati.

Plotting from data:
\begin{center}
\vspace{2ex}
\begin{figure}[!ht]
    \centering
    \begin{tikzpicture}
        \begin{axis}[
            title={Temperature dependence of CuSO\(_4\cdot\)5H\(_2\)O solubility},
            xlabel={Temperature [\textcelsius]},
            ylabel={Solubility [g per 100 g water]},
            xmin=0, xmax=100,
            ymin=0, ymax=120,
            xtick={0,20,40,60,80,100},
            ytick={0,20,40,60,80,100,120},
            legend pos=north west,
            ymajorgrids=true,
            grid style=dashed,
            ]

            \addplot[
                color=blue,
                mark=square,
                ]
                coordinates {
                (0,23.1)(10,27.5)(20,32)(30,37.8)(40,44.6)(60,61.8)(80,83.8)(100,114)
                };
                \legend{CuSO\(_4\cdot\)5H\(_2\)O}
                
            \end{axis}
        \end{tikzpicture}
    \caption{Grafico di solubilità del solfato di rame in base alla temperatura.}
    \label{plt:1}
\end{figure}

\end{center}
\newpage


\section{Conclusioni}
Qui trovano spazio eventuali note dell'operatore riguardanti aspetti procedurali e la sua valutazione dei risultati precedentemente elaborati, in riferimento agli obiettivi previsti dall'esperienza. 
\'E importante segnalare eventuali anomalie riscontrate nei confronti della metodica utilizzata, nonchè i passaggi che hanno causato difficoltà (in particolare sotto il profilo di sicurezza).

Ultima sezione della relazione, forse è la più importante insieme ad obbiettivo e procedimento. In questa parte si devono trarre le conclusioni dell'esperimento. 
\begin{itemize}
    \item Esito (riuscito-fallito);
    \item Risultato (sensato-assurdo), confrontando l'esito con quelli riportati in letteratura o tramite conti ricavati dalla letteratura;
    \item Osservazioni tue personali sui passaggi che potresti aver sbagliato o che magari ritieni di aver fatto nella maniera migliore rispetto alle indicazioni del prof, spiegare e motivare.
\end{itemize}
\section{Bibliografia}
Non sepre è richiesta ma è buona norma citare le fonti che si sono usate per stendere un qualsiasi tipo di elaborato scritto (ad eccezione del materiale del docente).
Questa sezione permette di raccogliere i siti e gli articoli da cui si è preso spunto o le informazioni (soprattutto quelle dei cenni teorici). 

Può anche servire nel caso si volessero recuperare le fonti per studiare o nel caso ci fosse un dibattito in merito a quanto scritto.

\info[inline]{Esempio di citazione.\autocite{einstein}}

\printbibliography %Prints bibliography %può servire o no, dipende da te
\section*{\quad \   Frasi H e P}%remove "\quad \   " if you use bibliography

Questa sezione è un'alternativa, nel caso non si vogliano riportare tutte le frasi P e H nelle tabelle dei reagenti si possono fare qui.
Vengono riportati i seguenti link, \href{https://it.wikipedia.org/wiki/Indicazioni_di_pericolo_H}{\textcolor{blue}{H}} e \href{https://it.wikipedia.org/wiki/Consigli_P}{\textcolor{blue}{P}}, che riguardano le frasi P e H in modo da avere un elenco ordinato.

Questo template incorpora anche un piccolo file che si occupa di riportare le frasi tramite un comando, basta inserire il numero e in automatico verrà riportato il testo in modo da rendere il file \LaTeX\ più snello.

\begingroup
\begin{multicols}{2}
\scriptsize
\subsection*{Pericoli fisici}
\begin{itemize}
    \item \Hphrase{H200}
    \item \Hphrase{H240}
\end{itemize}
\subsection*{Pericoli per la salute}
\begin{itemize}
    \item \Hphrase{H315}
    \item \Hphrase{H318}
    \item \dots
\end{itemize}
\subsection*{Consigli di prudenza di carattere generale}
\begin{itemize}
    \item \Pphrase{P101}
    \item \dots
\end{itemize}
\end{multicols}
\endgroup


 %non serve più se le aggiungi in tabella

\section*{link utili}
\begin{footnotesize}
\begin{multicols}{2}
\begin{itemize}
    \item \href{https://molview.org/}{\textcolor{blue}{MolView}: sito ottimo per avere immagini 2D e 3D dei composti chimici;}
    \item \href{https://echa.europa.eu/it/regulations/clp/clp-pictograms}{\textcolor{blue}{ECHA}: sito europeo per l'etichettatura e registrazione delle sostanze chimiche;}
    \item \href{https://echa.europa.eu/it/information-on-chemicals/cl-inventory-database}{\textcolor{blue}{ECHA}: per quanto riguarda le informazioni dei composti e la loro classificazione in europa;}
    \item \href{https://pubchem.ncbi.nlm.nih.gov/}{\textcolor{blue}{PubChem}: ottimo sito per informazioni generali dei composti chimici;}
    \item \href{http://www.chemspider.com/}{\textcolor{blue}{ChemSpider}: sito simile a PubChem;}
    \item \href{https://go.drugbank.com/}{\textcolor{blue}{DrugBank}}
    \item \href{https://www.ebi.ac.uk/chebi/init.do}{\textcolor{blue}{Chemical Entities of Biological Interest (ChEBI)}}
    \item \href{https://www.ebi.ac.uk/chembl/}{\textcolor{blue}{ChEMBL Database}}
    \item \href{https://commonchemistry.cas.org/}{\textcolor{blue}{CAS Common Chemistry}}
    \item \href{https://dguv.de/corona/index.jsp}{\textcolor{blue}{DGUV}}
    \item \href{https://www.rcsb.org/}{\textcolor{blue}{RCSB PDB}: Proteine DataBank}
    \item \href{https://www.efsa.europa.eu/it}{\textcolor{blue}{EFSA}: European Food Safety Authority}
    \item \href{http://eawag-bbd.ethz.ch/}{\textcolor{blue}{EAWAG BBD/PPS}: Biocatalysis/ Biodegradation Database;}
    \item \href{https://sinu.it/}{\textcolor{blue}{SINU}: Società Italiana di Nutrizione;}
    \item \href{https://www.ars.usda.gov/}{\textcolor{blue}{ARS}: Agricultural Research Service;}
    \item \href{https://www.nist.gov/}{\textcolor{blue}{NIST}: National Istitute of Standards and Technology;}
    \item \href{https://py-chemist.com/mol_2_chemfig/home}{\textcolor{blue}{Mol2chemifig}: sito per creare immagini e meccanismi di molecole facilmente e poi convertirli in codice LaTex per il pacchetto chemfig;}
    \item \href{https://www.rsc.org/merck-index}{\textcolor{blue}{The Merck Index Online}}: For over 120 years The Merck Index has been regarded as the most authoritative and reliable source of information on chemicals, drugs and biologicals;
    \item \href{http://libgen.is/}{\textcolor{blue}{Library Genesis}}: sito per scaricare articoli e libri, formato database;
    \item \href{https://it.z-lib.org/}{\textcolor{blue}{Z Library}}: sito per scaricare articoli e libri, formato più accessibile con immagini di copertina.
    \item \href{https://www.uniprot.org/}{\textcolor{blue}{UniProt}}: the mission of UniProt is to provide the scientific community with a comprehensive, high-quality and freely accessible resource of protein sequence and functional information.
\end{itemize}
\end{multicols}
\end{footnotesize}%link utili, elimina
\newpage
\section*{Esempi}
\subsection*{Immagini}

\begin{figure}[h]
    \centering
    \begin{subfigure}[]{0.5\linewidth}
    \centering
    \includegraphics[scale=0.2]{img/pittogrammi/Acute toxicity.png}
    \caption{Acute toxicity}
    \end{subfigure}
    \begin{subfigure}[]{0.5\linewidth}
    \centering
    \includegraphics[scale=0.2]{img/pittogrammi/Explosive.png}
    \caption{Explosive}
    \end{subfigure}
    \caption{Due immagini incolonnate al centro della pagina con posizione h definita da te.}
    \label{fig:1}
\end{figure}

\textit{Lorem ipsum dolor sit amet, consectetur adipiscing elit. Maecenas eget nisl elementum, pretium libero suscipit, interdum tellus. Praesent luctus commodo massa, vel molestie augue laoreet ac. Phasellus sodales auctor erat eu porta. Donec at volutpat nunc. Phasellus pretium, eros vitae cursus ornare, turpis massa egestas sem, ut varius ipsum metus nec mi. Integer ut odio erat. Interdum et malesuada fames ac ante ipsum primis in faucibus. Sed finibus, tortor in tristique iaculis, odio tortor finibus nisi, nec varius felis justo eu tortor. Interdum et malesuada fames ac ante ipsum primis in faucibus. Donec ultrices in ante vel imperdiet. Ut porttitor consectetur sodales. Vestibulum ante ipsum primis in faucibus orci luctus et ultrices posuere cubilia curae; Nullam non neque quis tortor convallis pulvinar id vel libero.}

\begin{figure}[!h]
\floatbox[{\capbeside\thisfloatsetup{capbesideposition={right,top},capbesidewidth=0.45\textwidth}}]{figure}[\FBwidth]
{\caption{didascalia dell'immagine centrale con didascalia a fianco.}}
{\includegraphics[width=0.1\textwidth]{img/acbenz.png}}
{\label{fig:2}}
\end{figure}

\textit{Aliquam erat nulla, suscipit id dignissim molestie, eleifend quis tortor. Pellentesque tempus egestas orci, sit amet eleifend elit condimentum id. Etiam id nisi velit. Etiam mauris nisl, facilisis sed egestas nec, tempor a est. In eget lacus vitae velit volutpat maximus. Vestibulum maximus arcu sit amet lacus maximus consequat. Duis sodales libero non metus finibus, eget ornare sem consequat.}

\begin{wrapfigure}{r}{0.3\textwidth} %this figure will be at the right
    \centering
    \includegraphics[scale=0.5]{img/acbenz.png}
    \caption{Didascalia dell'immagine sulla destra circondata da testo.} 
    \label{fig:3}
\end{wrapfigure}

\textit{Nullam vel lorem porttitor, convallis ex eget, condimentum velit. Integer sed sem aliquet, elementum ipsum in, rutrum elit. Vestibulum in arcu eget odio egestas varius. Vivamus lacus augue, dignissim at arcu a, consequat iaculis leo. Nam sit amet varius tellus. Nulla tempor velit nibh, et tincidunt diam porta vel. Mauris sit amet erat ut neque vehicula ultrices et eu sem. Phasellus pellentesque ultricies sapien. Curabitur at sodales mauris, maximus auctor mi. Vestibulum semper mauris in euismod rutrum. Fusce pellentesque in mi ac euismod.}

\textit{Nam gravida magna ut volutpat placerat. Pellentesque habitant morbi tristique senectus et netus et malesuada fames ac turpis egestas. Sed consequat justo condimentum metus bibendum mattis. Etiam id euismod ante. Nam sit amet ex libero. Proin id mauris at neque pellentesque accumsan eu ut ligula. Vivamus sodales magna sed risus faucibus tincidunt in eu felis.}

\begin{figure}
    \centering
    \includegraphics[scale=0.2]{img/pittogrammi/Acute toxicity.png}
    \includegraphics[scale=0.2]{img/pittogrammi/Explosive.png}
    \caption{Figure affiancate al centro della pagina nella posizione decisa da LaTex}
    \label{fig:4}
\end{figure}

\textit{Vivamus sit amet suscipit turpis, at eleifend risus. Vestibulum ante ipsum primis in faucibus orci luctus et ultrices posuere cubilia curae; Curabitur eget pharetra est, in mattis erat. Aenean pharetra finibus posuere. Nullam ipsum lacus, molestie nec eros eu, lacinia facilisis augue. Aenean dapibus rhoncus mi ut vulputate. Mauris commodo ultricies nulla egestas porttitor.}

\begin{figure}[h]
    \centering
    \includegraphics{img/buretta.jpg}
    \caption{Buretta}
    \label{fig:5}
\end{figure}

\newpage
\subsection*{Formule chimiche}
\begin{figure}[!ht]
    \chemfig{O=H}
\end{figure}

\begin{figure}[!ht]
    \chemfig{A-[1]B-[7]C}
    \caption{To define chemical formulae you can use units that define the angles}
\end{figure}

\begin{figure}[!ht]
    \chemfig{A-[:50]B-[:-25]C}
    \caption{Absolute angles}
\end{figure}


\begin{figure}[!ht]
    \chemfig{A-[::50]B-[::-25]C}
    \caption{Relative angles}
\end{figure}


\begin{figure}[!ht]
    \chemfig{A*5(-B=C-D-E=)}
    \caption{Regular polygons}
\end{figure}

\begin{figure}[!ht]
    \chemfig{A*5(-B=C-D)}
    \caption{Incomplete rings are also possible}
\end{figure}

\begin{figure}[!ht]
    \chemfig{H-C(-[2]H)(-[6]H)-C(=[1]O)-[7]H}
    \caption{Branched molecule}
\end{figure}

\begin{figure}
    \chemfig{A*6(-B=C(-CH_3)-D-E-F(=G)=)}
    \caption{Branched ring}
\end{figure}

\begin{figure}[!ht]
{\huge 
    \setchemfig{atom sep=2em,bond style={line width=1pt,red,dash pattern=on 2pt off 2pt}}  
    \chemname
    {\chemfig{H-C(-[2]H)(-[6]H)-C(=[1]O)-[7]H}}    
    {Ethanal}
}
\end{figure}

\newpage
\subsection*{Molecular orbital diagrams}
\begin{figure}[!ht]
        \begin{modiagram}
        \atom{left}{1s, 2s, 2p}
        \end{modiagram}
    \caption{First molecular orbital diagrams}
\end{figure}

\begin{figure}[!ht]
\begin{modiagram}
 \atom{right}{
    1s = { 0; pair} ,
    2s = { 1; pair} ,
    2p = {1.5; up, down }
 }


 \atom{left}{
    1s = { 0; pair} ,
    2s = { 1; pair} ,
    2p = {1.5; up, down }
 }
 \end{modiagram}
 \end{figure}

\begin{figure}[!ht]
 \begin{modiagram}
 \atom{left}{1s}
 \atom{right}{1s={;up}}
 \molecule{
    1sMO={0.75;pair,up}
  }
\end{modiagram}
\end{figure}

\begin{figure}[!ht]
\begin{modiagram}
 \atom{left}{
      1s, 2s, 2p = {;pair,up,up}
  }
  \atom{right}{
      1s, 2s, 2p = {;pair,up,up}
  }
  \molecule{
      1sMO, 2sMO, 2pMO = {;pair,pair,pair,up,up}
  }
\end{modiagram}
\end{figure}

\begin{figure}[!ht]
\begin{modiagram}
 \atom{left}{1s}
 \atom{right}{1s={;up}}
 \molecule{
    1sMO={;pair,up}
 }
 \draw[<-,shorten <=8pt,shorten >=15pt,blue]
 (1sigma*) --++(2,1) node {anti-bonding MO};
\end{modiagram}
\end{figure}

\newpage
\subsection*{Flowchart}
Ecco invece un esempio di flowchart che potrebbe essere utile in certi procedimenti particolarmente articolati.
\begin{figure}[!ht]
    \begin{center}
        \begin{tikzpicture}[node distance=2cm] % per guida -> https://it.overleaf.com/learn/latex/LaTeX_Graphics_using_TikZ%3A_A_Tutorial_for_Beginners_(Part_3)%E2%80%94Creating_Flowcharts
            \node (start) [startstop] {Inizio};
            \node (in1) [io, below of=start] {Ingresso};
            \node (pro1) [process, below of=in1] {Processo 1};
            \node (dec1) [decision, below of=pro1,  yshift=-0.5cm] {Decsione 1};
            \node (pro2a) [process, below of=dec1, yshift=-0.5cm] {Process 2a};
            \node (pro2b) [process, right of=dec1, xshift=2cm] {Process 2b};
            \node (out1) [io, below of=pro2a] {Output};
            \node (stop) [startstop, below of=out1] {Stop};
            
            \draw [arrow] (start) -- (in1);
            \draw [arrow] (in1) -- (pro1);
            \draw [arrow] (pro1) -- (dec1);
            \draw [arrow] (dec1) -- node[anchor=east] {yes} (pro2a);
            \draw [arrow] (dec1) -- node[anchor=south] {no} (pro2b);
            \draw [arrow] (pro2b) |- (pro1); %per fare linea segmentata
            \draw [arrow] (pro2a) -- (out1);
            \draw [arrow] (out1) -- (stop);
        \end{tikzpicture}
    \end{center}
\end{figure}

\newpage
\subsection*{Note e appunti vari}
Lorem ipsum dolor sit amet, consectetur adipiscing elit. Sed eget velit ullamcorper, convallis neque eget, tempus leo. Aenean eget est ornare, mattis turpis sit amet, vestibulum ante.\unsure{Change this!} Integer quis molestie arcu. Duis sem felis, posuere ut ante vitae, lobortis tincidunt dolor. Ut aliquet nunc sed lorem vehicula, ac tristique velit venenatis. Aliquam feugiat interdum magna, luctus sodales velit porta et. Nunc varius lorem nec varius malesuada. Praesent ante nisi, ultrices et venenatis sed, commodo vitae eros.\change{Change this!}

Pellentesque consectetur malesuada lectus, ut faucibus diam egestas ac. Aenean porttitor at libero a venenatis. Morbi sollicitudin, leo sed pellentesque facilisis, lacus diam lobortis tellus, sit \info{This can help me in chapter seven!}amet vulputate turpis sem id ipsum. Nunc ac aliquet mi, non porta quam. Maecenas auctor pulvinar sodales. Suspendisse eget mi arcu. Mauris quis nulla sit amet risus dapibus eleifend sed eget purus. Pellentesque libero nunc, \improvement{This really needs to be improved!\newline\newline What was I thinking?!}congue vitae mi sit amet, lobortis faucibus ante. Vestibulum cursus, neque quis auctor laoreet, sapien risus vestibulum orci, a rutrum tellus nisl quis ligula.

Sed non erat metus. Donec aliquet ex non neque sodales pretium. Aliquam eu sapien elit. Aliquam finibus felis et neque elementum, laoreet tempus urna auctor. Vivamus ac congue elit, vitae volutpat nulla. Sed pretium in lorem eget porttitor. Etiam interdum euismod odio, quis sollicitudin tellus rutrum et. Proin consequat, risus at consequat elementum, turpis elit tincidunt tellus, eu finibus dui mi vel eros. Donec in nulla tortor. Maecenas mollis consectetur erat sed elementum. Pellentesque habitant morbi tristique senectus et netus et malesuada fames ac turpis egestas. Nunc ultrices enim ut risus scelerisque, sed ultrices nibh congue. Mauris volutpat, elit vel sagittis consequat, lorem sapien iaculis sapien, id faucibus purus eros ut mauris. Vestibulum ornare elementum pretium.

Nunc non ante suscipit, dictum justo nec, dapibus elit. Proin lacinia leo fermentum dui bibendum, ac pellentesque felis malesuada. Integer sollicitudin tempor varius. Quisque sed magna rhoncus, ultricies nisi a, sollicitudin quam. Curabitur ut tempor lectus. Donec iaculis condimentum vehicula. In convallis ac sapien vel aliquam. Orci varius natoque penatibus et magnis dis parturient montes, nascetur ridiculus mus. Suspendisse ac eleifend dolor, at scelerisque urna. Quisque ac facilisis erat. Etiam accumsan risus sed molestie pulvinar. Orci varius natoque penatibus et magnis dis parturient montes, nascetur ridiculus mus. Proin non turpis non felis pretium convallis.
\improvement[inline]{The following section needs to be rewritten!}

Sed finibus pellentesque diam, et sagittis tortor ultrices ut. Orci varius natoque penatibus et magnis dis parturient montes, nascetur ridiculus mus. Suspendisse volutpat ullamcorper dui in cursus. Proin ullamcorper neque posuere porttitor rutrum. Suspendisse ac tortor justo. Vivamus convallis ligula at lacus commodo, eu mattis nulla vehicula. Vestibulum interdum leo et volutpat rhoncus.
\thiswillnotshow{This is hidden since option `disable' is chosen!}

E ora aggiungo una lista delle note.
\listoftodos[Notes to address] %parte di esempi per immagini o formule, elimina pure dopo aver scritto la relazione.
\end{document}
