\section*{link utili}

Piccola raccolta di link e pagine utili per scrivere la relazione.
\begin{footnotesize}
\begin{multicols}{2}
\begin{itemize}
    \item \href{https://molview.org/}{\textcolor{blue}{MolView}}: sito ottimo per avere immagini 2D e 3D dei composti chimici;
    \item \href{https://echa.europa.eu/it/regulations/clp/clp-pictograms}{\textcolor{blue}{ECHA-clp-pictograms}}: sito europeo per l'etichettatura e registrazione delle sostanze chimiche;
    \item \href{https://echa.europa.eu/it/information-on-chemicals/cl-inventory-database}{\textcolor{blue}{ECHA-cl-inventory-databas}}: per quanto riguarda le informazioni dei composti e la loro classificazione in europa;
    \item \href{https://pubchem.ncbi.nlm.nih.gov/}{\textcolor{blue}{PubChem}}: PubChem is the world's largest collection of freely accessible chemical information. Search chemicals by name, molecular formula, structure, and other identifiers. Find chemical and physical properties, biological activities, safety and toxicity information, patents, literature citations and more;
    \item \href{http://www.chemspider.com/}{\textcolor{blue}{ChemSpider}}: ChemSpider is a free chemical structure database providing fast text and structure search access to over 100 million structures from hundreds of data sources.
    \item \href{https://go.drugbank.com/}{\textcolor{blue}{DrugBank}}: Search our knowledge base for drug interactions, pharmacology, chemical structures, targets, metabolism, and more. Download limited datasets, free for academic and non-commercial researchers;
    \item \href{https://www.ebi.ac.uk/chebi/init.do}{\textcolor{blue}{Chemical Entities of Biological Interest (ChEBI)}}: Chemical Entities of Biological Interest (ChEBI) is a freely available dictionary of molecular entities focused on ‘small’ chemical compounds;
    \item \href{https://www.ebi.ac.uk/chembl/}{\textcolor{blue}{ChEMBL Database}}: ChEMBL is a manually curated database of bioactive molecules with drug-like properties. It brings together chemical, bioactivity and genomic data to aid the translation of genomic information into effective new drugs;
    \item \href{https://commonchemistry.cas.org/}{\textcolor{blue}{CAS Common Chemistry}}: CAS Common Chemistry is an open community resource for accessing chemical information. Nearly 500,000 chemical substances from CAS REGISTRY® cover areas of community interest, including common and frequently regulated chemicals, and those relevant to high school and undergraduate chemistry classes. This chemical information, curated by our expert scientists, is provided in alignment with our mission as a division of the American Chemical Society;
    \item \href{https://dguv.de/corona/index.jsp}{\textcolor{blue}{DGUV}}
    \item \href{https://www.rcsb.org/}{\textcolor{blue}{RCSB PDB}}: Il Protein Data Bank è un archivio per dati di struttura in 3-D di proteine e acidi nucleici. Questi dati, ottenuti soprattutto grazie alla cristallografia ai raggi X o alla spettrografia NMR, depositati da biologi e biochimici di tutto il mondo, sono di pubblico dominio e sono accessibili gratuitamente;
    \item \href{https://www.efsa.europa.eu/it}{\textcolor{blue}{EFSA European Food Safety Authority}}: Portale dedicato contenente informazioni esaurienti sulle nostre valutazioni del rischio: dalla ricezione di un mandato o di un fascicolo all’adozione dell’atto scientifico finale. Cliccate su questa sezione per verificare lo stato di avanzamento delle valutazioni scientifiche e per consultare dati, studi, ordini del giorno e verbali delle riunioni nonché informazioni sui nostri esperti;
    \item \href{http://eawag-bbd.ethz.ch/}{\textcolor{blue}{EAWAG BBD/PPS}}: Biocatalysis/ Biodegradation Database;
    \item \href{https://sinu.it/}{\textcolor{blue}{SINU}}: La Società Italiana di Nutrizione Umana (SINU) è una Società scientifica senza scopo di lucro che riunisce gli studiosi e gli esperti di tutti gli ambiti legati al mondo della nutrizione;
    \item \href{https://www.ars.usda.gov/}{\textcolor{blue}{ARS}}: Agricultural Research Service;
    \item \href{https://www.nist.gov/}{\textcolor{blue}{NIST}}: National Istitute of Standards and Technology;
    \item \href{https://py-chemist.com/mol_2_chemfig/home}{\textcolor{blue}{Mol2chemifig}}: sito per creare immagini e meccanismi di molecole facilmente e poi convertirli in codice LaTex per il pacchetto chemfig;
    \item \href{https://www.rsc.org/merck-index}{\textcolor{blue}{The Merck Index Online}}: For over 120 years The Merck Index has been regarded as the most authoritative and reliable source of information on chemicals, drugs and biologicals;
    \item \href{http://libgen.is/}{\textcolor{blue}{Library Genesis}}: sito per scaricare articoli e libri, formato database;
    \item \href{https://it.z-lib.org/}{\textcolor{blue}{Z Library}}: sito per scaricare articoli e libri, formato più accessibile con immagini di copertina.
    \item \href{https://www.uniprot.org/}{\textcolor{blue}{UniProt}}: the mission of UniProt is to provide the scientific community with a comprehensive, high-quality and freely accessible resource of protein sequence and functional information.
    \item \href{https://www.webelements.com/}{\textcolor{blue}{webelements}}: Pagina dedicata alla tavola periodica e ai suoi elementi.
    \item  \href{http://www.chm.bris.ac.uk/motm/motm.htm}{\textcolor{blue}{The Molecule of the Month}}: This is one of the longest running chemistry webpages on the internet. Each month since January 1996 a new molecule has been added to the list on this page, which makes this one of the longest running Chemical websites on the internet! The links will take you to a page at one of the Web sites at a University Chemistry Department or commercial site in the UK, the US, or anywhere in the world, where useful (and hopefully entertaining!), information can be found about a particularly interesting molecule.
    \item \href{https://chemistry.stackexchange.com/}{\textcolor{blue}{Chemistry-stackexchange}}
    \item \href{https://ctan.org/pkg/annotate-equations}{\textcolor{blue}{annotate-equations package}}
\end{itemize}
\end{multicols}
\end{footnotesize}
