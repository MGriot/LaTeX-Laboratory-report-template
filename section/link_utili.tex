\section*{link utili}
\begin{footnotesize}
\begin{multicols}{2}
\begin{itemize}
    \item \href{https://molview.org/}{\textcolor{blue}{MolView}: sito ottimo per avere immagini 2D e 3D dei composti chimici;}
    \item \href{https://echa.europa.eu/it/regulations/clp/clp-pictograms}{\textcolor{blue}{ECHA}: sito europeo per l'etichettatura e registrazione delle sostanze chimiche;}
    \item \href{https://echa.europa.eu/it/information-on-chemicals/cl-inventory-database}{\textcolor{blue}{ECHA}: per quanto riguarda le informazioni dei composti e la loro classificazione in europa;}
    \item \href{https://pubchem.ncbi.nlm.nih.gov/}{\textcolor{blue}{PubChem}: ottimo sito per informazioni generali dei composti chimici;}
    \item \href{http://www.chemspider.com/}{\textcolor{blue}{ChemSpider}: sito simile a PubChem;}
    \item \href{https://go.drugbank.com/}{\textcolor{blue}{DrugBank}}
    \item \href{https://www.ebi.ac.uk/chebi/init.do}{\textcolor{blue}{Chemical Entities of Biological Interest (ChEBI)}}
    \item \href{https://www.ebi.ac.uk/chembl/}{\textcolor{blue}{ChEMBL Database}}
    \item \href{https://commonchemistry.cas.org/}{\textcolor{blue}{CAS Common Chemistry}}
    \item \href{https://dguv.de/corona/index.jsp}{\textcolor{blue}{DGUV}}
    \item \href{https://www.rcsb.org/}{\textcolor{blue}{RCSB PDB}: Proteine DataBank}
    \item \href{https://www.efsa.europa.eu/it}{\textcolor{blue}{EFSA}: European Food Safety Authority}
    \item \href{http://eawag-bbd.ethz.ch/}{\textcolor{blue}{EAWAG BBD/PPS}: Biocatalysis/ Biodegradation Database;}
    \item \href{https://sinu.it/}{\textcolor{blue}{SINU}: Società Italiana di Nutrizione;}
    \item \href{https://www.ars.usda.gov/}{\textcolor{blue}{ARS}: Agricultural Research Service;}
    \item \href{https://www.nist.gov/}{\textcolor{blue}{NIST}: National Istitute of Standards and Technology;}
    \item \href{https://py-chemist.com/mol_2_chemfig/home}{\textcolor{blue}{Mol2chemifig}: sito per creare immagini e meccanismi di molecole facilmente e poi convertirli in codice LaTex per il pacchetto chemfig;}
    \item \href{https://www.rsc.org/merck-index}{\textcolor{blue}{The Merck Index Online}}: For over 120 years The Merck Index has been regarded as the most authoritative and reliable source of information on chemicals, drugs and biologicals;
    \item \href{http://libgen.is/}{\textcolor{blue}{Library Genesis}}: sito per scaricare articoli e libri, formato database;
    \item \href{https://it.z-lib.org/}{\textcolor{blue}{Z Library}}: sito per scaricare articoli e libri, formato più accessibile con immagini di copertina.
    \item \href{https://www.uniprot.org/}{\textcolor{blue}{UniProt}}: the mission of UniProt is to provide the scientific community with a comprehensive, high-quality and freely accessible resource of protein sequence and functional information.
\end{itemize}
\end{multicols}
\end{footnotesize}