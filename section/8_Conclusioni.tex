\section{Conclusioni}
Qui trovano spazio eventuali note dell'operatore riguardanti aspetti procedurali e la sua valutazione dei risultati precedentemente elaborati, in riferimento agli obiettivi previsti dall'esperienza. 
\'E importante segnalare eventuali anomalie riscontrate nei confronti della metodica utilizzata, nonchè i passaggi che hanno causato difficoltà (in particolare sotto il profilo di sicurezza).

Ultima sezione della relazione, forse è la più importante insieme ad obbiettivo e procedimento. In questa parte si devono trarre le conclusioni dell'esperimento. 
\begin{itemize}
    \item Esito (riuscito-fallito);
    \item Risultato (sensato-assurdo), confrontando l'esito con quelli riportati in letteratura o tramite conti ricavati dalla letteratura;
    \item Osservazioni tue personali sui passaggi che potresti aver sbagliato o che magari ritieni di aver fatto nella maniera migliore rispetto alle indicazioni del prof, spiegare e motivare.
\end{itemize}