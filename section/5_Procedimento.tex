\section{Procedimento}
Riferisce la procedura  operativa seguita. Spesso si completa con un disegno schematico dell'attrezzatura utilizzata, quando è utile per descrivere le istruzioni di assemblaggio della stessa.
Qui bisogna spiegare in maniera sintetica ma esaustiva tutti i passaggi svolti durante l'esperimento.

Sono ammessi anche dei commenti o delle osservazioni, se hanno un senso. Nel caso di particolari non menzionati nella procedura ma osservati durante l'esperienza come l'aumento di temperatura in una reazione che impedirebbe di tenere in mano una provetta da saggio.

\info[inline]{es. Aggiungere la lega di di Devarda. \textit{Attenzione! Dopo l'aggiunta il contenuto della provetta raggiunge alte temperature, meglio svolgere l'operazione vicino ad un bancone e con una scarabattola per posare la provetta.}}


Un altra cosa che puoi aggiungere è un flowchart che riassuma i passaggi, questo può servire anche per aver chiaro il procedimento e eventualmente studiare.

