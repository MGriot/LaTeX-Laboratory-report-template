\section{Procedimento}
Riferisce la procedura  operativa seguita. Spesso si completa con un disegno schematico dell'attrezzatura utilizzata, quando è utile per descrivere le istruzioni di assemblaggio della stessa. Qui devi spiegare in maniera sintetica ma esaustiva tutti i passaggi da te svolti durante l'esperimento.

Sono ammessi anche dei commenti o delle osservazioni, se hanno senso. Magari ti sei accorto che una reazione è particolarmente esotermica e la provetta diventa troppo calda da tenere in mano e allora puoi scrivere in corsivo, o con altri stratagemmi, perfar capire che questo parte esula dal procedimento ma che è un consiglio per la buona riuscita dell'esperimento.

es.
Aggiungere la lega di di Devarda. 
\textit{Attenzione! Dopo l'aggiunta il contenuto della provetta raggiunge alte temperature, meglio svolgere l'operazione vicino ad un bancone e con una scarabattola per posare la provetta.}


Un altra cosa che puoi aggiungere è un flowchart che riassuma i passaggi, questo può servirti a te in prima batutta per aver chiaro il procedimento e eventualmente studiare.

