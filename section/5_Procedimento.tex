\section{Procedure}
Reports the operating procedure followed. It is often completed with a schematic drawing of the equipment used, when it is useful to describe the assembly instructions of the same.
Here you need to explain in a synthetic but exhaustive way all the steps taken during the experiment.

Comments or observations are also allowed, if they make sense. In the case of details not mentioned in the procedure but observed during the experiment, such as the increase in temperature in a reaction that would prevent holding a test tube in your hand.

\info[inline]{e.g. Add Devarda's alloy. \textit{Warning! After the addition the content of the test tube reaches high temperatures, it is better to carry out the operation near a counter and with a crucible to place the test tube.}}


Another thing you can add is a flowchart that summarizes the steps, this can also be useful to have a clear procedure and possibly study.
