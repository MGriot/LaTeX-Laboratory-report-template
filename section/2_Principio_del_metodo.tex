\section{Principle of the method}
Sometimes it is referred to as Summary or Theoretical principle.

It refers to all the theoretical principles on which the experiment is based and how they were used to achieve the set objectives. Here you should put all the theoretical concepts necessary to understand the report, you need to explain everything that deserves to be explained. If you have already written reports, or the report is part of a macro group of reports, you can think of omitting some parts that could be repeated and streamline the whole thing a bit.

An example can be:
\begin{description}
	\item[Stoichiometry] The relationship between the relative quantities of substances taking part in a reaction or forming a compound, typically a ratio of whole integers.
	\item[Atomic mass] The mass of an atom of a chemical element expressed in atomic mass units. It is approximately equivalent to the number of protons and neutrons in the atom (the mass number) or to the average number allowing for the relative abundances of different isotopes. 
\end{description} 

Alternatively, use the command \verb|\subsection{}| to indicate the various points.
