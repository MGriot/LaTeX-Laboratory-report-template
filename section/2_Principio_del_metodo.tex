\section{Principio del metodo}
A volte è indicato come Sommario o Principio teorico.

Fa riferimento a tutti i principi teorici su cui si basa l'esperienza e a come essi si sono utilizzati per raggiungere gli obbiettivi prefissati. Qui vanno messsi tutti i concetti teorici necessari a comprendere la relazione, bisogna spiegare tutto quello che merita di essere spiegato. Se si sono già scritte delle relazioni, o la relazione fa parte di un macro gruppo di relazioni, si può pensare di omettere alcune parti che potrebbero ripetersi e snellire un po' il tutto.

Un esempio può essere:
\begin{description}
	\item[Stoichiometry] The relationship between the relative quantities of substances taking part in a reaction or forming a compound, typically a ratio of whole integers.
	\item[Atomic mass] The mass of an atom of a chemical element expressed in atomic mass units. It is approximately equivalent to the number of protons and neutrons in the atom (the mass number) or to the average number allowing for the relative abundances of different isotopes. 
\end{description} 

Oppure usare il comando \verb|\subsection{}| per indicare i vari punti.
